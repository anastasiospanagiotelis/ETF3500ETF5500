\documentclass{article}
\usepackage{amsmath}
\usepackage{bm}
\usepackage{multicol}
\addtolength{\oddsidemargin}{-1in}
\addtolength{\evensidemargin}{-1in}
\addtolength{\textwidth}{2in}

\addtolength{\topmargin}{-1in}
\addtolength{\textheight}{2in}
\title{Tutorial Solutions:  Matrices and Distance}
\author{High Dimensional Data Analysis}
\date{}
\begin{document}
	
Work in groups of 2 people:
\begin{enumerate}
	\item Consider the \textbf{age} and \textbf{height} of both you and the other person (you are allowed to lie about these).  
	\begin{enumerate}
		\item Compute the Euclidean distance between you and the other person for these two variables.
		
		{\em Euclidean distance between LeBron James (height 203cm, age 33) and Tom Cruise (height 170 cm, age 56) is 40.22.  Notice that units of measurement affect calculation.}
		\item Compute the Manhattan distance between you and the other person for these two variables.
		
		{\em Manhattan distance between LeBron James (height 203cm, age 33) and Tom Cruise (height 170 cm, age 56) is 56}
	\end{enumerate}
	\item Select from the following list the types of cuisines the ones that you enjoy
	\begin{multicols}{2}
		\begin{itemize}
			\item Chinese food
			\item Indian food
			\item Italian food
			\item Japanese food
			\item Lebanese food
			\item Mexican food
			\item Thai food
			\item British food
		\end{itemize}
	\end{multicols}
	\begin{enumerate}
		\item Compute a Jaccard similarity between you and the other person with regards to your taste in food.
		
		{\em The Jaccard similarity between someone who only likes Chinese food and someone who likes Chinese Thai, Italian and Japanese is 1/4.} 
		\item Compute a Jaccard distance between you and the other person with regards to your taste in food.
		
		{\em The Jaccard similarity between someone who only likes Chinese food and someone who likes Chinese Thai, Italian and Japanese is 1-1/4=3/4.}
	\end{enumerate}
	\item How would you define a distance between you and the other person that takes into account height, age and food preference.
	
	{\em One idea might be to add Manhattan(Euclidean) distance and Jaccard distance together}
\end{enumerate}
\end{document}