\documentclass{article}
\addtolength{\oddsidemargin}{-.875in}
\addtolength{\evensidemargin}{-.875in}
\addtolength{\textwidth}{1.75in}

\addtolength{\topmargin}{-.875in}
\addtolength{\textheight}{1.75in}
\title{Tutorial:  Introduction and Preliminary Data Analysis}
\author{High Dimensional Data Analysis}
\date{Week 1}
\begin{document}
\maketitle
\section{Motivation}
\begin{enumerate}
	\item Describe the role that preliminary data analysis plays in the overall statistical analysis of a dataset. Give examples to illustrate your answer.

    {\em Preliminary data analysis is very important. It helps identify data problems (such as outliers, missing values,
	etc.). Moreover, it helps you get a “feel” for your data and also helps to identify potential relationships. For
	example, a boxplot will help identify if spending is different according to gender. A scatterplot or histogram
	and descriptive statistics can help identify outliers either visually or due to strange ranges and large differences
	between the mean and median}

	\item Think  of  a  problem  from  business  or  another  discipline  where  some  concepts  of	groups is important.   In
	particular think of a case where:
	\begin{enumerate}
		\item Groups are unknown and the aim is to classify individuals into similar groups.
        
        {\em A department store may wish to know the main types of customers, for example one important demographic may be women aged over 55 who make many purchases with a low value, while another important demographic may be men aged 35-55 who make a small number of high value purchases. Cluster analysis allows the data to speak for themselves in determining these groups. Separate marketing strategies can then be developed for each group or cluster.}
		
		\item Groups are known and the aim is test whether the differences between groups is significant.
        
        {\em A supermarket chain may have data on sales of different classes of products (e.g. fruit, vegetables, cleaning products etc.) across different stores (e.g. Caulfield, City, St. Kilda). Local advertising campaigns may be depend on whether there are statistically significant differences in spending across the stores for any of the products. This can be tested using Multivariate Analysis of Variance (MANOVA).}
		
		\item Groups are known for some observations and the aim is to predict group membership for the remaining
		observations.
		
		{\em An online store may have information on the browsing behaviour of existing customers who purchased a specific product and those who did not purchase a specific product. The company may want to predict whether a new customer will ultimately belong to the purchasing group or the non-purchasing group.}
	\end{enumerate}
\end{enumerate}
\section{Measurement}
Think of an example of a non-metric variable and
metric variable.  For each variable answer the following:
\begin{enumerate}
	\item Is the variable measured on a	nominal, ordinal or ratio scale?
	
	{\em Mode of transport is a nominal variable while stock returns are ratio variables}
    
    \item What would be a good summary or plot to use to get an idea about this data?
    
    {\em The proportion individuals who use each mode of transport while the mean, mode and median all give
    some idea of the central tendency of stock returns.}
\end{enumerate}
\section{Introduction to R}
\begin{enumerate}
	\item Open R Studio on your workstation and load the pacakge ggplot2.  Once you have familiarised yourself with R, install R, R studio and the add-on package ggplot2 on to your own laptop.
	\item The aim of the following exercise is to replicate some of the preliminary data analysis carried out in the first lecture.  You will need to load the dataset Beer.RData which can be downloaded from Moodle.
    \begin{enumerate}
    \item Produce a histogram of the price  variable.  Use the function {\em qplot} for this.
    \item Do you identify any outliers in the data?
    \item Produce a cross tab of beer rating against origin.  Use the function {\em table} for this.
    \end{enumerate}
    {\em This example will be covered again in the next tutorial with more detailed answers provided.}
\end{enumerate}
\end{document}